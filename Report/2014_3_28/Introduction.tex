\section{Introduction}
\label{sec:Introduction}
Nowadays, as the concept of Internet of Things (IoT) gradually becomes more and
more important, IP cameras on road can be seen in many countries.
These cameras are connected to the Internet and responsible for some
surveillance applications such as traffic monitoring or crime prevention.
However, the installation of these cameras are often around a small area.
That is to say, image data collected from these cameras might be correlated to
each other.
Therefore, we argue in this paper that we can make use of this correlation and
try to reduce the radio resource used to transmit the image data.

More specifically, if we consider two cameras allocated at a crossroad, we can
analyze their correlation by letting one camera as a I-frame transmitter while
the other as a P-frame transmitter.
Under this assumption, higher correlation level means that we can reduce more
transmission bits when serving the network.
In this paper, we will work on how to serve the wireless multimedia sensor
network (WMSN) better via the overhearing source coding.
We generate the testing images by a 3-D modeling software and these images are
passed through the H.264 reference software to estimate the correlation matrix.
Based on the correlation matrix, we further present a scheduling algorithm to
serve the network so that we can efficiently reduce the total transmission time
needed.

The rest of this paper is organized as follows:
Section~\ref{sec:NetworkScenario} describes our multimedia network scenario and
Scetion~\ref{sec:ProblemFormulation} states our problem formulation.
Section~\ref{sec:SchedulingAlgorithm} solves this problem and
Section~\ref{sec:SimulationResults} shows the simulation results.
Finally, Section~\ref{sec:Conclusion} concludes our work.