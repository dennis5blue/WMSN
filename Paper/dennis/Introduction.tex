\section{Introduction}
\label{sec::Introduction}

%Nowadays, as the concept of Internet of Things (IoT) gradually becomes more and
%more important, IP cameras on road can be seen in many countries.
%These cameras are connected to the Internet and responsible for some
%surveillance applications such as traffic monitoring or crime prevention.
%However, the installation of these cameras are often around a small area.
%That is to say, images collected from these cameras might be correlated to
%each other.
%Therefore, we argue in this paper that we can make use of this correlation and
%try to reduce the usage of radio resource for transmitting these images.

Many machine-to-machine (M2M) applications are characterized by a large amount of
data to transport over a rather limited amount of wireless resources. 
While for some applications the amount of data produced by each machine is not
huge and the delay sensitivity is rather low, for applications such as multimedia
surveillance networks the requirement on communication is demanding.
%
Fortunately, since such wireless surveillance cameras are typically deployed with
overlapping view angles, the images or videos captured by individual cameras exhibit
correlation that can potentially be leveraged for bandwidth-efficient reporting of
the collected data.

In this paper, we investigate the problem of correlated data gathering from a set
of cameras deployed in a city. It is required that cameras periodically send back
the collected images back to the aggregator (e.g. base station) through direct 
wireless communications (e.g. LTE or WiMAX).
Since there might be multiple cameras deployed in a neighborhood area to provide
different perspectives of the area, we exploit the capability of {\em transmission overhearing}
among cameras.
%{\em transmission overhearing} and allow a camera to reference images (e.g. as an
%I-frame) transmitted by others for encoding local image (e.g. as a P-frame).
If a camera can overhear transmissions from nearby cameras, it can reference the
image (e.g. as an I-frame) and perform {\em dependent coding} to reduce the 
amount of bits required to encode its image (e.g. as a P-frame).
%Besides, we also assume that all cameras in $V$ can perform an overhearing
%source coding scheme.
%Specifically, the camera
%That is, they are able to gather others transmission and use the gathered image
%can reference the image overheard from nearby camera
%for motion prediction. 
Clearly, if the reference image is highly correlated with the target
image, the compression ratio will be high.
%
We propose a {\em correlation-aware scheduling algorithm} to determine the order of transmissions
for all cameras based on their locations and the correlation of collected images.
To evaluate the proposed algorithm, we resort to a $3$D modeling software to generate
quasi-realistic city views for all cameras and use H.264 MVC reference software to
encode collected images.
Evaluation results show the proposed scheduling algorithm to outperform
baseline approaches, motivating further investigation along this direction.
%
%More specifically, if we consider two cameras allocated at a crossroad, we can
%analyze their correlation by letting one camera as an I-frame transmitter while
%the other as a P-frame transmitter.
%Under this assumption, higher correlation level means that we can reduce more
%transmission bits when serving the network.
%In this paper, we will analyze the correlation between cameras in a city and
%work on how to serve the wireless multimedia sensor network (WMSN) better via
%the overhearing source coding.
%We generate the testing images by a $3$D modeling software and these images are
%passed through the H.264 reference software to estimate their correlation.
%Based on the correlation between cameras, we further present a scheduling
%algorithm to serve the network so that we can efficiently reduce the total
%encoded bits for transmission required.

\ignore{
The rest of this paper is organized as follows:
%Section~\ref{sec::DifferentialEncoding} describes the idea of correlation-based
%differential encoding in WMSN while
Section~\ref{sec::ProblemFormulation} states the network scenario and problem formulation.
Section~\ref{sec::SchedulingAlgorithm} presents the proposed scheduling and
Section~\ref{sec::OverhearingExperiments} shows the experiment settings and
results.
Finally, Section~\ref{sec::Conclusion} concludes the paper. 
}