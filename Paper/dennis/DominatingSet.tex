\subsection{Dominating Set Problem}
\label{sec::DominatingSet}
{\color{blue}
Recall that our goal in this paper is to determine a proper transmission
schedule.
We here claim that the transmission scheduling problem can be seen as choosing
part of the cameras as {I-frame} transmitters (broadcasters) while the other
cameras are {P-frame} transmitters (listeners).
The transmission schedule is thus simple, we let all the broadcasters to be
transmitted before listeners so that the later scheduled listeners are able to
reference from those previous scheduled broadcasters for reducing its encoded
bits.
Therefore, what we interested in is how to select broadcasters to let those
listeners have the capability to reduce the most encoded bits.
Since the listeners are able to reduce their encoded bits but the broadcasters
cannot, one intuitive idea for radio resource conservation is to select fewest
broadcasters where the transmission range of those broadcasters can cover the
whole network (since a listener can only reduce encoded bits when it can
overhear a broadcaster's transmission).

We now refer to the \emph{Minimum Dominating Set} problem for selecting the
fewest broadcasters in the surveillance network.
In some applications, the \emph{Minimum Dominating Set} problem is formulated
as a integer programming problem~\cite{MDS}.
Therefore, the broadcasters selection problem can be written as:
\begin{align}
\text{minimize} &~~\sum_{i=1}^N \sum_{j=1}^N \alpha_{ii}H(X_j|X_i),\nonumber \\
\text{subject to} &~~A \vec{\alpha} \succeq k \vec{1}, \nonumber \\
 &~~\alpha_{ii} \in \{ 0,1 \},
\label{eq::MDS}
\end{align}
where $A$ is an adjacency matrix indicates the overhearing capability ($i-j$
entry $=1$ if camera $v_i$ is able to overhear camera $v_j$) and $\vec{\alpha}$
is a vector where its $i^{th}$ component is $\alpha_{ii}$.
The ides of Problem~\eqref{eq::MDS} is that we want to select a subset of
broadcasters so that they can have the minimum encoded bits if other listeners
reference from those broadcasters.
The first constraint of Problem~\eqref{eq::MDS} is to protect that all cameras
can have at least $k$ candidate reference broadcasters (covered by the
transmission range of at least $k$ broadcasters).
By relaxing the second constraint of Problem~\eqref{eq::MDS} from ${\alpha_{ii} 
\in \{0,1\}}$ to ${0 \leq \alpha_{ii} \leq 1}$, we can get a linear programming
problem as:
\begin{align}
\text{minimize} &~~\sum_{i=1}^N \sum_{j=1}^N \alpha_{ii}H(X_j|X_i),\nonumber \\
\text{subject to} &~~A \vec{\alpha} \succeq k \vec{1}, \nonumber \\
 &~~0 \leq \alpha_{ii} \leq 1,
\label{eq::relaxedMDS}
\end{align}
Therefore, Problem~\eqref{eq::relaxedMDS} can be solved easily as a
conventional linear programming problem.
After obtaining the value of $\vec{\alpha}$, camera $v_i$ is selected as a
broadcaster with probability $\alpha_{ii}$ and the rest listeners will choose
the best broadcasters as its reference camera.

Note that $k=1$ in the conventional \emph{Minimum Dominating Set} problem since
the goal of \emph{Minimum Dominating Set} problem is to select the fewest nodes
to cover the whole graph.
However, $k=1$ is not suitable in our paper because if a camera $v_i$ can only
select its reference frame from one candidate broadcaster, it happens that the
broadcaster is not correlated with camera $v_i$, causing that the system
performance becomes lower. 
Therefore, we here give the motivation to increase the value of $k$ in
Problem~\eqref{eq::relaxedMDS} and the influence of $k$ will be shown in our
evaluation results.

}
{\color{OliveGreen}
%We also try use a greedy algorithm to solve the broadcaster selection problem
%in a surveillance cameras network.
%First, we need to mention that any maximal independent set (MIS) is a
%dominating set if they consider the same graph, and constructing a MIS can be
%done by a greedy algorithm~\cite{MISisDS}.
%In reference~\cite{MIS}, the authors generate the MIS by an adaptive greedy
%function.
%In this paper, we refer to the idea in~\cite{MIS} but slightly change the
%greedy objective
We also try to use graph theory to solve the broadcaster selection problem in
surveillance cameras network.
Note that the idea of $k$ introduced in Problem~\eqref{eq::relaxedMDS} is
similar to the $k$-tuple dominating set problem.
The authors in~\cite{ICGA} proposed an \emph{ICGA} algorithm which is able to
generate a $k$-tuple dominating set based on centralized decision.
The \emph{ICGA} algorithm constructs a dominating set first and iteratively add
node whose dominator neighbors is less than $k$ into the dominating set by a
greedy criteria.
We here first construct a dominating set $\mathcal{D}$ by the algorithm
proposed in~\cite{MIS}, and apply the \emph{ICGA} algorithm for generating the
$k$-tuple dominating set $\mathcal{D}_k$.
The overall procedure of the algorithm is summarized in
Algorithm~\ref{alg::kTupleDS}, and the performance will also be compared in
our evaluation results.

}

\begin{algorithm}[t]
\caption{Constructing a $k$-tuple dominating set $\mathcal{D}_k$
\label{alg::kTupleDS}}
\begin{algorithmic}[1]
\State Construct a dominating set $\mathcal{D}$ using GRASP~\cite{MIS}
\State $\mathcal{D}_k \gets \mathcal{D}$
\While{There has a node whose number of dominator neighbors is less than $k$}
\State $\mathcal{F} \gets$ All the nodes whose number of dominator neighbors is
less than $k$
\State Add the node that dominates the largest number of nodes in $\mathcal{F}$
to $\mathcal{D}$ 
\State $\mathcal{D}_k \gets \mathcal{D}$ 
\EndWhile
\end{algorithmic}
\end{algorithm} 